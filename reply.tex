%%%%                                                 %%%%%
%%%%%%%%%%%%%%%%%%%%%%%%%%%%%%%%%%%%%%%%%%%%%%%%%%%%%%%%%%%
%\documentclass[prd,amsmath,amssymb,showpacs,superscriptaddress,nofootinbib]{revtex4-1}
\documentclass[a4paper]{revtex4-2}
\usepackage{indentfirst}
\usepackage{hyperref}
\usepackage{multirow}
\usepackage{latexsym,bm,amsmath,amssymb}
\usepackage{graphicx}
\usepackage{epsfig}
\usepackage{subfigure}
%\usepackage{cite}
\usepackage{color}
\usepackage[top=1in, bottom=1in, left=1.25in, right=1.25in]{geometry}
\usepackage{slashed}
%\usepackage{tikz}
%\usepackage{cite}
\usepackage{fancyhdr}
\usepackage{slashed}
\usepackage{makecell,rotating,multirow,diagbox}
\usepackage{tikz}
\usepackage{float}
%\usepackage{hyperref}
\usepackage{appendix}
\usepackage{setspace}
\usepackage{geometry}
\usepackage{graphics}
\newcommand{\red}{\color{red}}
\newcommand{\blue}{\color{blue}}

%\renewcommand{\thefootnote}{\fnsymbol{footnote}}


\begin{document}
\begin{center}{\Large Reply to the referee}
\end{center}
Manuscript ID: CPC-2025-0888

``Studies on quark-mass dependence of the $N^*(920)$ pole from $\pi N$ $\chi$PT amplitudes'' By Xu Wang, Kai-Ge Kang, 
Qu-Zhi Li, Zhiguang Xiao and Han-Qing Zheng.

We are grateful that the referee raised some helpful question for us to improve our work.
Below, we address the referees’ comments one by one and provide clarifications and explanations
for changes made in the manuscript. Our responses and modifications in the main text are
highlighted in blue.

\section*{Response to Reviewer 1's comments. }

In this article the authors have studied the quark mass dependence of the N*(920) subthreshold pi N resonance in the S11 channel. The article can be easily followed, however, I found it a bit misleading because of these reasons:

\emph{1) The authors state in the introduction that the existence of this
resonance has been confirmed by the Roy-Steiner method and cite the article [8].
However, in [8] it is clearly said that even if they find the pole in their
calculation, the physical interpretation is far from obvious, so that they kind
of point out that this resonance might be model dependent. Then, are there
experimental or lattice signatures of this resonance? If it is not the case as
it seems, are the pole parameters result of a model dependent calculation? Are
there other methods where the pi – N interaction is studied and no related pole
is observed? Is this resonance for example obtained in some QCD sum rule
calculation or different methods? Is the existence of this resonance really
tight to consequences of model independent assumptions in the pi - N
scattering?}

\emph{In my opinion the authors should explain with more caution whether the existence of this pole is a model independent result or not.}


Response: 

{\blue Roy-Steiner equations are derived from the fundamental principles of the
  Quantum field theory and scattering theory, making them well-accepted as
  model-independent. A resonance pole discovered within the validity domain by
  solving these equations is always considered to exist model independently. A
  famous example is the $\sigma$ resonance which is finally confirmed to exist
  using this method. The $N^*(920)$ pole is obtained by solving these equations
  numerically and thus its existence is model independent. There are other model
  dependent studies that also suggested the existence of a similar pole, for
  example, Eur. Phys. J. A 43, 83–105 (2010) by D\"oring et.al.. However, since
  these studies are model dependent, their findings can not establish the
  existence of this pole. In fact, some authors even attribute this pole
  contribution to the left-hand cut contribution (such as in Nucl. Phys. A 829,
  170–209 (2009) and the previous reference by D\"oring et. al.) until the
  Roy-Steiner equation result confirms its existence. See [9,10] of the new
  version for recent reviews.

  In [8], the statement ``the physical interpretation is far from obvious'' does
  not mean that the existence of this resonance pole is model dependent.
  Instead, it highlights the need for more investigation into the physics
  underlying this pole, such as whether it plays a role in chiral dynamics
  within hadron interactions. This question also serves as the motivation for
  our work.
\\
}

\emph{2) Related also with the previous item I think the authors should clarify and motivate the physical relevance of the N*(920) pole parameters. For example, which physical observable can be or has been measured which can be dependent or related to the N*(920) pole parameters?.}

{\blue Response: For normal resonance, the pole position always lies above the threshold of the decay channel, the real part of the pole position represents the mass of the state and the imaginary part is related to the half decay width of the state. However, since $N^*(920)$ pole lies below the threshold with large imaginary part, it can not be directly observed by the experiment. But in fact its effect is already implicitly present in the observed phase shift of $\pi N$ scattering. It was demonstrated that the left-hand cut contribution to the phase shift is always negative, and thus there should be a resonance pole there to reconcile the left-hand cut contribution to the phase shift to produce the experimental result.~[3,4].
}

\emph{3) The way in which the quark mass dependence is studied is not very much rigorous. For example, the authors use a simple parameterization for the nucleon mass, mN=800 MeV+ mpi. However, clearly the nucleon mass can be studied in BchPT, that they use to study the quark mass dependence of the N*(920) pole. Given that there are LQCD data for mN at different pion masses, see for example, recent RQCD data,  JHEP 05, 035 (2023), the authors should show that with their formalism they can describe well LQCD data available for baryons before studying the N*(920) resonance, in particular for the nucleon.
}

{\blue Response: In our main study, we use the pion mass dependence of the parameters
  $m_N$, $g_A$ and $F_\pi$ in Eq.(26) which are just the results within the
  framework of B$\chi$PT, not the ruler approximation. The ruler approximation
  $m_N=800 MeV+m_\pi$ is just used as an alternative approximation which is
  consistent with the lattice results. In this approximation, the $m_N$ is found
  to be consistent with the lattice result in a large $m_\pi$ range as reported
  in Fig.(28) (page 63) of Phys.Rep.625 (2016).  In our paper, we only consider 
  two-flavor case, but the reference JHEP 05, 035 (2023) pointed out by the referee 
includes also the third flavor. So, it may not be applicable for the present
work.}
\\[1.5cm]
Reviewer: 2

Comments to the Author
\emph{In this work, the authors investigate the quark-mass dependence of the  $ N^*(920) $ pole using unitarized chiral perturbation theory ($\chi$PT) amplitudes up to $ O(p^2) $ and $ O(p^3) $. They conclude that, as the quark mass is varied, the pole eventually crosses the $ u $-channel cut and moves onto an adjacent Riemann sheet, in qualitative agreement with predictions from the linear sigma model. The study addresses an interesting and timely topic. However, the presentation could be significantly improved to enhance clarity and readability. Before I can recommend the manuscript for publication, the following points should be addressed:}\\

\emph{1) Since the K-matrix method is employed to unitarize the $\chi$PT amplitudes, it is not strictly correct to refer to the approach simply as “$\chi$PT” in the title without mentioning unitarization. After all, unitarization is a nonperturbative scheme that lies beyond the scope of perturbative theory.}\\

{\blue Response: in the title $\chi$PT is modified to ``unitarized $\chi$PT'' }
\\[.3cm]

\emph{2) To address the chiral extrapolation at $\mathcal{O}(p^2)$ and $\mathcal{O}(p^3)$, several low-energy constants (LECs) are introduced. However, the manuscript does not discuss the quark mass dependence of these LECs. It would be helpful to clarify why their variations with the quark mass (pion mass) are not taken into account.}\\

{\blue Response: In chiral perturbation, the basic principle is to do the calculations according to the powers of $m_\pi$ and $p$, thus the low energy coefficients by definition are $m_\pi$ independent. In other words, if a coefficient depends on $m_\pi$ it should also be expanded with respect to $m_pi$ to leave the $m_\pi$ independent expansion coefficients to be the LECs of used by the $\chi$PT. So, there is no need to consider the $m_\pi$ dependence of the LECs.
This is also the usual practice in the $\chi$PT community, see e.g. Phys. Rev. Lett. 100,152001.}
\\[.3cm]

\emph{3) In this manuscript, many values of the LECs are taken from the literature. However, upon checking, I find that at least some of them are not cited from the original papers in which these LECs were actually determined. For example, below Eq. (27), Refs. [31] and [32] are cited for the values of  $d_{16}$ and $\bar{l}_4$. However, a careful reading of these two papers shows that the quoted values are themselves taken from earlier references. Proper citation is important not only for giving appropriate credit, but also for allowing readers to more easily trace the original sources. I have only checked these two cases; similar citations throughout the manuscript should be carefully and thoroughly verified.}\\

{\red Need to check the reference}


\emph{4) The authors present results obtained within several different schemes for comparison. The $ O(p^2) $ and $ O(p^3) $ results shown in Fig. 3 appear to be the “baseline” results that the paper aims to address. The $ O(p^2) $ parameters are taken from Ref. [3], while the $ O(p^3) $ results are taken from Ref. [33] and are referred to as the Yao set. The tree–Li and loop–Li results originate from the linear $\sigma$-model. The $ O(p^3)$-WI08 results are taken from Ref. [34]. In addition, a scheme constrained by lattice QCD results and phenomenological models is presented in Fig. 5. I have several suggestions regarding the presentation:}\\


\emph{4.a) The $ O(p^3) $ results in Fig. 3 and the $ O(p^3)$-(Yao) results in Fig. 4 are in fact the same. It would be better to unify the notation—specifically, whether to keep the “Yao” suffix—to avoid potential confusion. My suggestion is to omit the suffix for the baseline results that the paper focuses on, and retain explicit suffixes only for results only for comparison. Similarly, the two results shown in Fig. 5 should also be introduced with appropriate suffixes.}\\

{\blue We thank the referee for careful proofreading. The notation in these figures is now corrected to be consistent.}
\\

\emph{4.b) In $\pi N$ studies, a crucial issue is whether the $\Delta$ resonance is included as an explicit degree of freedom. My understanding is that the Yao set includes the $\Delta$, whereas the WI08 set does not. This important distinction should be stated explicitly. For the remaining schemes—for example, the baseline $ O(p^2) $ results and the schemes shown in Fig. 5—it should also be clarified whether the $\Delta$ is treated explicitly or not. Moreover, the authors should comment on whether different treatments of the $\Delta$ are expected to lead to similar or qualitatively different results.}\\


\emph{4.c) Closely related to point (b) is the question of whether the LECs determined in schemes with and without an explicit $\Delta$ are used in a mixed manner. With the current presentation, this is not immediately clear to the reader, and clarification would be helpful.}\\


{\blue Response to both 4.b) and 4.c): In our paper, the $\Delta$ state is not explicitly included. In Yao and  WI08 papers, both the cases with explicit $\Delta$ and without are discussed, and  we have chosen the parameters in the cases without explict $\Delta$ in our study.  \red need check}


\emph{4.d) In the BChPT with $\Delta$, does the $m_\pi$-depedence of the $\Delta$ mass would affect the final  results?}\\ 


\emph{4.e) In Section 3, the value $ c_1 = -0.841,\mathrm{GeV}^{-1} $ is stated twice: once in Eq. (25) and again in the text below Eq. (27). Similarly, the value $ c_1 = -1.22,\mathrm{GeV}^{-1} $ is also stated twice, both below Eq. (27) and in Eq. (29). I suggest first presenting all analytical expressions in a unified manner, and then, for each scheme, listing all corresponding parameter values clearly in a single place.}\\

{\blue We have modified the definition as the referee suggested.}
\\

\emph{5) In Fig. 5, the $ m_\pi $-dependence of $ g_A $, $ F_\pi $, and $ m_N $ is addressed by several lattice-inspired models. It would be preferable to present their $ m_\pi $-dependence explicitly. For example, Fig. 2 could be separated into three subfigures, showing $ m_N $, $ g_A $, and $ F_\pi $ as functions of $ m_\pi $, respectively.}\\

{\blue We have separated figure 2 into three subfigures as the referee suggested. The lattice-inspired models are not the main consideration in this paper, and serves just as a cross check for the main results. So we directly taken the values from the figures of the fit results in the original papers, and we would refer the readers to the original papers for the figures.}

\emph{6) In the lines 13,14 of Page 6, it was stated ``..., we have sufficient confidence that...".  While the statement conveys the authors' conviction, emphasizing a claim based mainly on confidence can be seen as subjective. It is better to modify it.}\\

{\blue We have modified it to `` it is expected that''.}

7) Last but not least, the authors emphasize that the pole position eventually crosses the (u)-channel cut and moves onto an adjacent Riemann sheet. I wonder whether this feature has any important physical implications.

{\blue Response: We are also not clear whether this feature has any physical implications. As we stated, since this pole is previously unknown, any exploration of its possible conceptual understanding is valuable even if it does not have any physical consequence. Since perturbation theory may not be valid for large $m_\pi$, we are not clear where is the pole heading for  with larger $m_\pi$. Anyway, this study provides a reference for future lattice study with unphysical pion mass.
}

\end{document}
